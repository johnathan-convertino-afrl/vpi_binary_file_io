\begin{titlepage}
  \begin{center}

  {\Huge VPI\_BINARY\_FILE\_IO}

  \vspace{25mm}

  \includegraphics[width=0.90\textwidth,height=\textheight,keepaspectratio]{img/AFRL.png}

  \vspace{25mm}

  \today

  \vspace{15mm}

  {\Large Jay Convertino}

  \end{center}
\end{titlepage}

\tableofcontents

\newpage

\section{Usage}

\subsection{Introduction}

\par
This library provides two functions.
\begin{itemize}
\item read\_binary\_file(FILE\_NAME, VECTOR)
\item write\_binary\_file(FILE\_NAME, VECTOR)
\end{itemize}
\par
Each instance is a new instance, and will start reading the file from the start.
The vector has to be in size bytes from 1 to any number of bytes. Each function
returns the number of bytes read or writen. Z or X place in the vector indicates
bytes not available for read, or do not write these bytes for write. The read funciton
will return a negative number of bytes when the end of file is reached.

You can use the following for including the library in your project:

\begin{lstlisting}[language=XML]
src :
  files :
    - src/read_binary_file.c  : {file_type : cSource}
    - src/write_binary_file.c : {file_type : cSource}
    - src/binary_file_io.c    : {file_type : cSource}
    - src/binary_file_io.h    : {file_type : cSource, is_include_file : true}
    - src/read_binary_file.h  : {file_type : cSource, is_include_file : true}
    - src/write_binary_file.h : {file_type : cSource, is_include_file : true}
    - src/binary_file_io.sft  : {file_type : user}

lib :
  files :
    - lib_ringbuffer/build/libringBuffer.a : {file_type : user, copyto : .}

header :
  files :
    - lib_ringbuffer/ringBuffer.h : {file_type : cSource, is_include_file : true}

vpi:
  bin_file_io_vpi:
    filesets : [src, header]
    libs : [ringBuffer -L., pthread]

generate:
  gen_git:
    generator: git_pull
    parameters:
      repo_url: https://github.com/sparkletron/C89_pthread_ring_buffer.git
      repo_dir: lib_ringbuffer
      tag: release_v1.6.1
  gen_lib:
    generator: gen_cmake
    parameters:
      src_dir:  lib_ringbuffer
      cmake_args: ["-DCMAKE_POSITION_INDEPENDENT_CODE=ON"]

targets:
  default: &default
    description: Intergration default target for simulations.
    filesets: [lib, dep_gen]
    generate: [gen_git, gen_lib]
    vpi: [bin_file_io_vpi]
\end{lstlisting}

\subsection{Dependencies}

\par
The following are the dependencies of the cores.

\begin{itemize}
  \item fusesoc 2.X
  \item iverilog (simulation)
\end{itemize}

\input{src/fusesoc/depend_fusesoc_info.tex}

\section{Architecture}
\par
This VPI library provides two functions for the user to use during simulation, read\_binary\_file and write\_binary\_file. These will read and
write from binary files. These functions use ringbuffers and multithreading to seperate file I/O from the simulation so file access will not
slow down the simulation.
\par
The read\_binary\_file will read any binary file till it runs out of data. When it does, if it can not complete the word (one byte left, for say a 4 byte word output)
then the unused bytes for the aval/bval pairs are set to Z. Meaning in the simulation they will show up as Z, not 0 or 1. It will also assert the EOF (end of file) signal
from the core showing that this is the last of the data.
\par
The write\_binary\_file will write any binary data till it is given that is a 0 or a 1. Any bytes that contain a Z will not be written to the output file. This allows for
any file that is read to be written in a one to one manner.

Please see \ref{Module Documentation} for more information per target.

\section{Building}

\par
The all VPI binary file IO source files are written in C to target the VPI API from Verilog 2001. They should simulate in any modern simulation tool that has VPI support.
The library comes as a fusesoc packaged core and can be included in any other testbench. Be sure to make sure you have meet the dependencies listed in the previous section.

\subsection{fusesoc}
\par
Fusesoc is a system for building FPGA software without relying on the internal project management of the tool. Avoiding vendor lock in to Vivado or Quartus.
These cores, when included in a project, can be easily integrated and targets created based upon the end developer needs. The core by itself is not a part of
a system and should be integrated into a fusesoc based system. Simulations are setup to use fusesoc and are a part of its targets.

\subsection{Source Files}

\subsubsection{fusesoc\_info File List}
\begin{itemize}
\item src
	\begin{itemize}
	\item {'src/read\_binary\_file.c': {'file\_type': 'cSource'}}
	\item {'src/write\_binary\_file.c': {'file\_type': 'cSource'}}
	\item {'src/binary\_file\_io.c': {'file\_type': 'cSource'}}
	\item {'src/binary\_file\_io.h': {'file\_type': 'cSource', 'is\_include\_file': True}}
	\item {'src/read\_binary\_file.h': {'file\_type': 'cSource', 'is\_include\_file': True}}
	\item {'src/write\_binary\_file.h': {'file\_type': 'cSource', 'is\_include\_file': True}}
	\item {'src/binary\_file\_io.sft': {'file\_type': 'user'}}
	\end{itemize}
\item lib
	\begin{itemize}
	\item {'lib\_ringbuffer/build/libringBuffer.a': {'file\_type': 'user', 'copyto': '.'}}
	\end{itemize}
\item header
	\begin{itemize}
	\item {'lib\_ringbuffer/ringBuffer.h': {'file\_type': 'cSource', 'is\_include\_file': True}}
	\end{itemize}
\item tb
	\begin{itemize}
	\item {'tb/tb\_vpi.v': {'file\_type': 'verilogSource'}}
	\end{itemize}
\end{itemize}


\subsection{Targets} \label{targets}

\subsubsection{fusesoc\_info Targets}
\begin{itemize}
\item default
	\begin{itemize}
	\item[$\space$] Info: Intergration default target for simulations.
	\end{itemize}
\item sim
	\begin{itemize}
	\item[$\space$] Info: Test VPI file io.
	\end{itemize}
\item sim\_rand\_data
	\begin{itemize}
	\item[$\space$] Info: Test VPI file io with random data.
	\end{itemize}
\item sim\_8bit\_count\_data
	\begin{itemize}
	\item[$\space$] Info: Test VPI file io with count data.
	\end{itemize}
\end{itemize}


\subsection{Directory Guide}

\par
Below highlights important folders from the root of the directory.

\begin{enumerate}
  \item \textbf{docs} Contains all documentation related to this project.
    \begin{itemize}
      \item \textbf{manual} Contains user manual and github page that are generated from the latex sources.
    \end{itemize}
  \item \textbf{src} Contains source files for vpi binary file io.
  \item \textbf{tb} Contains test bench files.
\end{enumerate}

\newpage

\section{Simulation}
\par
A barebones test bench for iverilog is included in tb/tb\_vpi.v . This can be run from fusesoc with the following.
\begin{lstlisting}[language=bash]
$ fusesoc run --target=sim AFRL:vpi:binary_file_io:1.0.0
\end{lstlisting}

\newpage

\section{Code Documentation} \label{Module Documentation}

\begin{itemize}
\item \textbf{VPI BINARY FILE SOURCE, DOXYGEN}
\end{itemize}
The next section documents the library.

